%%%%%%%%%%%%%%%%%%%%%%%%%%%%%%%%%%%%%%%%%
% Friggeri Resume/CV
% XeLaTeX Template
% Version 1.0 (5/5/13)
%
% This template has been downloaded from:
% http://www.LaTeXTemplates.com
%
% Original author:
% Adrien Friggeri (adrien@friggeri.net)
% https://github.com/afriggeri/CV
%
% License:
% CC BY-NC-SA 3.0 (http://creativecommons.org/licenses/by-nc-sa/3.0/)
%
% Important notes:
% This template needs to be compiled with XeLaTeX and the bibliography, if used,
% needs to be compiled with biber rather than bibtex.
%
%%%%%%%%%%%%%%%%%%%%%%%%%%%%%%%%%%%%%%%%%

\documentclass[]{friggeri-cv} % Add 'print' as an option into the square bracket to remove colors from this template for printing
\usepackage{indentfirst}
\usepackage{bm} 
\setlength{\parindent}{2em}
\begin{document}

\header{个人}{简历}{方祯-西安电子科技大学} % Your name and current job title/field

%----------------------------------------------------------------------------------------
%	SIDEBAR SECTION
%----------------------------------------------------------------------------------------

\begin{aside} % In the aside, each new line forces a line break
\section{个人姓名}
方祯
\section{学历}
硕士
\section{毕业时间}
2015/04/01
\section{手机}
13679120182
\section{邮箱}
\href{mailto:fz1989fz@gmail.com}{fz1989fz@gmail.com}
\section{编程语言}
Python, C/C++
\section{求职意向}
研发工程师
\end{aside}
%----------------------------------------------------------------------------------------
%	EDUCATION SECTION
%----------------------------------------------------------------------------------------

\section{教育情况}

\begin{entrylist}
%------------------------------------------------
\entry
{2012--2015}
{硕士 {\normalfont 计算机体系结构}}
{西安电子科技大学}
{\emph{云计算资源调度组 导师: 马建峰}\\
主要研究与计算平台的资源分配,如虚拟机资源分配,虚拟机放置策略的研\\
究,任务的分配和任务的执行顺序等相关研究}
%------------------------------------------------
\entry
{2008--2012}
{学士 {\normalfont 计算机科学与技术教改班}}
{西安电子科技大学}
{本科期间多次获得学校奖学金}
%------------------------------------------------
\end{entrylist}

\section{专业技能}
\begin{itemize}
    \item 熟悉C/C++,python编程语言和面向对象编程,掌握基本的软件开发
    \item 熟悉Linux系统和openstack,具有良好的算法功底
    \item 阅读大量学术文献,具有良好的文献阅读能力和撰写能力
\end{itemize}

%----------------------------------------------------------------------------------------
%	WORK EXPERIENCE SECTION
%----------------------------------------------------------------------------------------

\section{项目经历}

\begin{entrylist}

%------------------------------------------------
\entry
{2013--至今}
{电磁协同云计算平台}
{西安电子科技大学}
{\bullet  \textbf{项目描述}\\
基于在openstack平台上开发自动部署电磁计算框架和执行电磁计算任务,主\\
目标支持动态部署,资源调度,迁移和灾备等要\\
\bullet \textbf{主要工作}
\begin{enumerate}
    \item 研究云计算平台资源和任务的调度问题。
    \item 负责整体后台任务处理系统的编写
    \item 虚拟机中,接受服务后台的控制指令,并返回执行状态。
    \item 处理输入,输出文件的中间传输问题和以及服务接口的设计。
\end{enumerate}}
%------------------------------------------------
\entry
{2014}
{交通信息采集项目}
{西安电子科技大学}
{\bullet \textbf{项目描述}\\
该项目是通过红绿灯上摄像头采集,调用厂家api获取红绿灯车辆经过信息。\\
由于厂家api无法提供存储服务,编写多路接收系统和存储,并将发送数据库\\
服务,将原有系统的收发能力提高了5倍\\
\bullet \textbf{主要工作}
\begin{enumerate}
    \item 负责接收程序和数据库服务程序的设计
    \item 多线程处理接收端处理摄像头发来的图片程序
    \item 将发送过来的图片和识别信息存储至本地存储,并将需要保存的信\\
        息发送至数据库存储端
\end{enumerate}}

%------------------------------------------------
\entry
{2013}
{享车网项目}
{享车网初期团队}
{\bullet  \textbf{项目描述} \\
享车网是一个定位于汽车"后服务"市场的互联网平台。整合线下汽车服务店资\\
源,致力于服务本地广大车主,提供高性价比的汽车综合服务。\\
\bullet \textbf{主要工作}
\begin{enumerate}
    \item 负责用户登陆模块的编写。
    \item 负责后台数据库车型模块的编写
    \item 负责商家和用户后台后端的编写。
\end{enumerate}}
%------------------------------------------------
\end{entrylist}

%----------------------------------------------------------------------------------------
%	AWARDS SECTION
%----------------------------------------------------------------------------------------

\section{获奖经历}

\begin{itemize}
    \item 2013西电hackday最佳作品奖
    \item 2013年研究生一等奖学金
    \item 2013晋级yahoo hackday决赛
    \item 2012年腾讯Qreboot大赛金奖
    \item 2012晋级腾讯马拉松决赛
    \item 2011ACM/ICPC北京赛区铜奖
    \item 2011ACM/ICPC上海赛区铜奖
    \item 2011校级一等奖学金
    \item 2010ACM/ICPC天津赛区铜奖
    \item 2010西安电子科技大学程序竞赛一等奖
    \item 2010西安电子科技大学数学建模校赛二等奖
    \item 2010校级二等奖学金
    \item 2009校级二等奖学金
\end{itemize}

%----------------------------------------------------------------------------------------
%	COMMUNICATION SKILLS SECTION
%----------------------------------------------------------------------------------------

\section{实践和比赛经历}

\begin{entrylist}
%------------------------------------------------

\entry
{2014}
{CSDN开源夏令营}
{西安电子科技大学}
{memcached的性能优化问题,目前正在入手和研究这一问题优化和解决办法}

\entry
{2013}
{yahoo北研hackday决赛}
{yahoo北研中心,北京}
{通过算法的初赛晋级决赛,决赛中编写《爱游》手机应用,记录用户的旅行的\\
照片,地点等信息。个人负责手机后台处理。}
%------------------------------------------------
\entry
{2013}
{西电hackday}
{西安电子科技大学}
{作品《机智路由》,主要是在路由器上添加webserver能提供认证上网,设备\\
控制等功能,个人主要负责后台处理}
%------------------------------------------------

\entry
{2012}
{微软imagineCup}
{西安电子科技大学}
{手机阅读类软件ReadOne开发,负责收集网上材料提供用户阅读,主要负责后\\
台推荐算法和用户后台排名的实现}

\entry
{2012}
{腾讯Qreboot应用大赛}
{西安电子科技大学}
{腾讯Qreboot语音机器人应用开发大赛,获取语音系统的输入,编写相应机器\\
人的反馈程序,最终作品《百变小呆》和《小Q背单词》分获金奖和优秀奖}
%------------------------------------------------

\entry
{2012}
{腾讯马拉松决赛}
{腾讯总部,深圳}
{通过算法的初赛晋级决赛,决赛中编写《街景涂鸦》手机应用,通过手机相机\\
的镜头获取街景,在此之上显示人们对此的评价。个人负责后台数据处理。}

\entry
{2010}
{MiniSQl解释器}
{西安电子科技大学}
{基于lex和yacc做了一个miniSQl的C语言解释器,实现了简单sql语法的分析\\
和后台分页存储的实现。个人负责词法分析,语法分析以及语法树的构建 }
%------------------------------------------------
\end{entrylist}

%----------------------------------------------------------------------------------------
%	INTERESTS SECTION
%----------------------------------------------------------------------------------------

%\section{个人评价}

%乐观积极,具有团队精神,喜欢钻研问题,找出优秀的解决方案。

%----------------------------------------------------------------------------------------
%	PUBLICATIONS SECTION
%----------------------------------------------------------------------------------------

%\section{publications}

%----------------------------------------------------------------------------------------


\end{document}
